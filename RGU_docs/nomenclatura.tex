\chapter{Nomenclatura}
%--------------------------------------------------------------------
\textcolor{red}{En desarrollo!!!!!}
\begin{description}[labelindent=1cm,labelwidth=2.25cm,align=left,leftmargin=3.45cm]  %no cambiar esta línea

\item[$REGEX$] \emph{Regular expressions}, secuencia de caracteres que conforman un patrón de búsqueda.

\item[$A$] \emph{Acumulador},registro en el que son almacenados temporalmente los resultados aritméticos e intermedios.

\item[$(A)$], se refiere al contenido del acumulador.

\item[$C$] \emph{Bandera de Acarreo},contiene un acarreo (0 o 1) del bit MSB luego de operaciones aritméticas y algunas operaciones de corrimiento y rotación.

\item[$CPU$] \emph{Central Unit Processing},Bit más significativo, el de más a la izquierda en la cadena de bits.

\item[$DI$] \emph{DI},Dirección de donde se obtiene la instrucción.

\item[$I$], bandera de interrupción.

\item[$M$], dirección efectiva del operando.

\item[$MSB$] \emph{Most Significant Bit},Bit más significativo, el de más a la izquierda en la cadena de bits.

\item[$N$], bandera de signo.

\item[$P$], apuntador de pila.

\item[$PC$], contador de programa.

\item[$S$], registro de estado.

\item[$V$], bandera de rebase.

\item[$[XX]$], dirección del operando o del contenido XX, [16] se refiere a la localización de memoria donde el número 16 está almacenado.

\item[$(XX)$], contenido de la localizacion XXXX, (0143) se refiere al contenido de la localización 0143.

\item[$V$], bandera de rebase.

\item[$Z$], bandera cero.


%\textbf{\item[$SysC$] \emph{Regular expressions}, Secuencia de caracteres que conforman un patrón de búsqueda, comúnmente en forma de strings.}
%	\item[$API$] \emph{Application Programming Interface}, especifica interoperación de componentes de software, normalmente en forma de librerías.
%	\item[$C10k$] \emph{Ten Thousand Connection}, es un acrónimo dado al problema que percibe un servidor al tener que procesar una gran cantidad de peticiones simultaneas, alrededor de 10 mil.
%	\item[$CMYK$] \emph{Cyan, Magenta, Yellow, blacK}, es el acrónimo que se le da al espacio de color que consiste en restar a un medio blanco los colores secundarios para conseguir un espacio RGB.
%	\item[$CSRF$] \emph{Cross-site request forgery}, es un tipo de inseguridad generado a partir de que un usuario en el cual el servidor confía empieza a transmitir comandos no permitidos.
%	\item[$DOM$] \emph{Document Object Model}, es un API que proporciona un conjunto de objetos para representar documentos HTML y XML, un modelo en como se pueden combinar y como se accede a ellos. El responsable del DOM es el W3C.
%	\item [$FPS$] \emph{Frames Per Second}, tasa a la cual se reproducen los cuadros de un video.
%	\item[$GUI$] \emph{Graphical User Interface}, programa informático que por medio de un conjuntos de objetos e imágenes le proporciona al usuario un entorno visual con el que puede interactuar con el programa.
%	\item[$HSI$] \emph{Hue, Saturation, Intensity}, espacio de color creado por la forma en que las personas perciben y describen el color. Se tiene el matiz que es similar al tono del color, la saturación que es cuanto de ese matiz está presente y la intensidad que representa que tan claro o brillante esta dicho color.
%	\item[$HTML$] \emph{Hypertext Markup Language}, lenguaje de marcado utilizado para la elaboración de páginas web.
%	\item[$HTTP$] \emph{Hypertext Transfer Protocol}, protocolo para sistemas de información en Internet.
%	\item[$JSON$] \emph{JavaScript Object Notation}, formato de intercambio de datos utilizado por herramientas JavaScript. Presenta objetos legibles sin la necesidad de utilizar XML.
%	\item[$MVC$] \emph{Model View Controller}, patrón de diseño en software que divide la funcionalidad del mismo en tres grandes módulos que trabajan en conjunto para dar la funcionalidad completa a la aplicación.
%	\item[$OS$] \emph{Operating System}, se refiere al software que administra hardware, recursos de software y presta servicios comunes a los programas de un computador.
%	\item[$RGB$] \emph{Red, Green, Blue}, espacio de color que se basa en la suma de colores primarios para representar el rango del espectro electromagnético visible.
%	\item[$URL$] \emph{Uniform Resource Locator}, es un tipo de URI pero que a los datos a los que este se refiere pueden variar en el tiempo. Están formados por una secuencia de caracteres, de acuerdo a un formato modélico y estándar, que designa recursos en una red.
%	\item[$URI$] \emph{Uniform Resource Identifier}, es una cadena de caracteres que identifica los contenidos de una red de manera unívoca.
%	\item[$W3C$] \emph{World Wide Web Consortium}, consorcio internacional que da recomendaciones para el uso de la World Wide Web.
%	\item[$XML$] \emph{Extensible Markup Language}, lenguaje de marcas desarrollado por el W3C, ampliamente utilizado para almacenar datos o codificar archivos de manera que sea legible tanto por personas como por un computador.
%	\item[$XSS$] \emph{Cross-site scripting}, tipo de inseguridad informática que permite a un tercero inyectar un código JavaScript en páginas frecuentadas por el usuario, evitando medidas de control como la política del mismo origen.
\end{description}
