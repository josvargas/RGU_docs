\documentclass{eieproyecto}
\usepackage{gensymb}
\usepackage{epstopdf}
\usepackage{verbatim}
\usepackage{listings}
\usepackage{color}
\usepackage{subcaption}
\definecolor{dkgreen}{rgb}{0,0.6,0}
\definecolor{gray}{rgb}{0.5,0.5,0.5}
\definecolor{mauve}{rgb}{0.58,0,0.82}

\definecolor{lightgray}{rgb}{0.95, 0.95, 0.95}
\definecolor{darkgray}{rgb}{0.4, 0.4, 0.4}
\definecolor{purple}{rgb}{0.65, 0.12, 0.82}
\definecolor{editorGray}{rgb}{0.95, 0.95, 0.95}
\definecolor{editorOcher}{rgb}{1, 0.5, 0} % #FF7F00 -> rgb(239, 169, 0)
\definecolor{editorGreen}{rgb}{0, 0.5, 0} % #007C00 -> rgb(0, 124, 0)
\usepackage[table,xcdraw]{xcolor}
\usepackage{upquote}
\usepackage{pgfgantt}
% CSS

%\usepackage{bera}% optional: just to have a nice mono-spaced font
\usepackage{xcolor}
\usepackage{float}
\restylefloat{table}



\usepackage{amsmath}
\usepackage{amsfonts} 
\usepackage{amssymb}
\usepackage{algorithm}
\usepackage[noend]{algpseudocode}
%\usepackage{algorithmic}

%Packages
\usepackage{xspace}
\usepackage{color}

%Colours
%ucr
\definecolor{celesteUCR}{RGB}{65,173,231}
\definecolor{amarilloUCR}{RGB}{249,234,100}
\definecolor{verdeUCR}{RGB}{81,186,69}
%
%tum
\definecolor{blueTUM}{RGB}{0,105,178}
%
\definecolor{redPRIS}{RGB}{180,0,0}
\definecolor{yellowPRIS}{RGB}{240,230,170}
\definecolor{grayPRIS}{RGB}{90,100,140}
\definecolor{blackPRIS}{RGB}{0,0,0}
%
\definecolor{prisred}{RGB}{180,0,0}
\definecolor{prisgrey}{RGB}{180,150,150}
\definecolor{darkgrey}{RGB}{50,60,70}
\definecolor{lightgrey}{RGB}{150,150,150}
\definecolor{labgrey}{RGB}{90,100,140}
\definecolor{labyellow}{RGB}{240,230,170}
\definecolor{prisgray}{RGB}{50,60,70}
\definecolor{prisblack}{RGB}{0,0,0}
\definecolor{prisblue}{RGB}{40,65,140}
\definecolor{ace1}{RGB}{0,150,215}
\definecolor{ace2}{RGB}{0,50,70}
\definecolor{ribBlue}{RGB}{0,100,150}

\definecolor{pris}{RGB}{180,0,0}
\definecolor{prisblue}{RGB}{40,65,140}
% \definecolor{prisgray}{RGB}{107,107,107}
\definecolor{labyellow}{RGB}{240,230,170}
\definecolor{lightgrey}{RGB}{150,150,150}

%Teknovae colors
\definecolor{teknovaeBlack}{RGB}{0,0,0}
\definecolor{teknovaeOrange}{RGB}{255,139,51}
\definecolor{teknovaeGreen}{RGB}{3,204,0}
\definecolor{teknovaeBlue}{RGB}{65,26,255}
%
\definecolor{blackTKN}{RGB}{0,0,0}
\definecolor{redTKN}{RGB}{255,139,51}
\definecolor{greenTKN}{RGB}{3,204,0}
\definecolor{blueTKN}{RGB}{65,26,255}

%Fonts
\DeclareFontFamily{T1}{pbk}{}
\DeclareFontFamily{T1}{pbks}{}
\DeclareFontShape{T1}{pbk}{m}{n}{<->s*pbkl}{}
\DeclareFontShape{T1}{pbks}{m}{n}{<->s*[0.7]pbkl}{}

%Names
\newcommand{\prislab}{{\fontfamily{pbk}\selectfont\textcolor{prisred}{PR}\textcolor{prisblack}{IS}}{\fontfamily{pbks}\selectfont\textcolor{prisgray}{-LAB}}\xspace}
\newcommand{\pris}{Pattern Recognition and Intelligent Systems Laboratory\xspace}
\newcommand{\prises}{Laboratorio de Investigación en Reconocimiento de Patrones y Sistemas Inteligentes\xspace}
\newcommand{\prisfull}{\prislab: \pris}
\newcommand{\prisfulles}{\prislab: \prises}

\definecolor{core1}{RGB}{75,95,115}
\definecolor{core2}{RGB}{220,60,60}
\newcommand{\core}{{\fontfamily{bookman}\sc \textcolor{core1}{C}\textcolor{core2}{o}\textcolor{core1}{re}}\xspace}

\definecolor{excess1}{RGB}{33,180,168}
\definecolor{excess2}{RGB}{190,0,60}
\newcommand{\excess}{{\fontfamily{bookman}\sc \textcolor{excess1}{E}\textcolor{excess2}{x}\textcolor{excess1}{cess}}\xspace}

\definecolor{essence1}{RGB}{90,140,50}
\definecolor{essence2}{RGB}{124,222,50}
\newcommand{\essence}{{\fontfamily{bookman}\sc \textcolor{essence1}{E}\textcolor{essence2}{ss}\textcolor{essence1}{ence}}\xspace}

\newcommand{\bend}{{\fontfamily{bookman}\sc \textcolor{excess2}{B}\textcolor{excess1}{e}\textcolor{excess1}{nd}}\xspace}
\newcommand{\hint}{{\fontfamily{bookman}\sc \textcolor{essence1}{H}\textcolor{essence2}{int}}\xspace}

% PRIS-Lab
% \newcommand{\pris}{Pattern Recognition and Intelligent Systems Laboratory\xspace}
% \newcommand{\prislab}{{\fontfamily{bookman}\sc\textcolor{prisred}{PR}\textcolor{black}{IS\textcolor{prisgray}{-lab}}}\xspace}
% \newcommand{\bcrn}{{\fontfamily{bookman}\sc\textcolor{prisred}{PR}\textcolor{black}{IS\textcolor{prisgray}{-lab}}}\xspace}
% \newcommand{\prisfull}{{\fontfamily{bookman}\sc\prislab}: \pris}
% \newcommand{\prislabfull}{\sc Laboratorio de Investigación en\\\textcolor{redPRIS}{Reconocimiento de Patrones}\\y Sistemas Inteligentes}
\newcommand{\aprs}{{\fontfamily{bookman}\sc APRS}\xspace}
\newcommand{\aprsfull}{{\fontfamily{bookman}\sc\aprs}: Advanced Pattern Recognition System\xspace}
\newcommand{\aprsfulles}{{\fontfamily{bookman}\sc\aprs}: Sistema Avanzado de Reconocimiento de Patrones\xspace}
%
\newcommand{\prissem}{{\fontfamily{bookman}\sc\textcolor{prisred}{PR}\textcolor{black}{IS\textcolor{black}{-Seminar}}}\xspace}
\newcommand{\coloquio}{\textit{Coloquio de Investigación}\xspace}
\newcommand{\ace}{{\fontfamily{bookman}\sc\textcolor{ace1}{A}\textcolor{ace2}{ce}}\xspace}

\newcommand{\sura}{{\fontfamily{bookman}\sc\textcolor{prisred}{S}\textcolor{black}{u}\textcolor{prisblue}{r}\textcolor{prisgray}{á}}\xspace}
\newcommand{\boknama}{{\fontfamily{bookman}\sc Boknama}\xspace}
\newcommand{\iriria}{{\fontfamily{bookman}\sc Iriria}\xspace}

%Networks
\newcommand{\rib}{{\fontfamily{bookman}\sc\textcolor{ribBlue}{RIB}}\xspace}
\newcommand{\ribfull}{\rib: Red de Investigación en Biocomputación\xspace}
\newcommand{\bcrn}{{\fontfamily{bookman}\sc\textcolor{ribBlue}{BcRN}}\xspace}
\newcommand{\bcrnfull}{\bcrn: Biocomputing Research Network\xspace}

\newcommand{\rider}{{\fontfamily{bookman}\sc \textcolor{orange!80!black}{Rider}}\xspace}
\newcommand{\riderfull}{\rider: Red de Investigación y Desarrollo en Eficiencia Energética y Tecnologías en Energía Renovable\xspace}

\newcommand{\ricc}{{\fontfamily{bookman}\sc \textcolor{red!80!black}{R}\textcolor{green!80!black}{I}\textcolor{blue!80!black}{C}\textcolor{prisgray}{C}}\xspace}
\newcommand{\riccfull}{\ricc: Red de Investigación en Computación Científica\xspace}
\newcommand{\scrn}{{\fontfamily{bookman}\sc \textcolor{red!80!black}{S}\textcolor{green!80!black}{C}\textcolor{blue!80!black}{R}\textcolor{prisgray}{N}}\xspace}
\newcommand{\scrnfull}{\scrn: Scientific Computing Research Network\xspace}

%Institutions
\newcommand{\arcoslab}{ARCOS-LAB\xspace}
\newcommand{\arcoslabfull}{\arcoslab: Autonomous Robots and Cognitive Systems Laboratory\xspace}

\newcommand{\clcar}{CLCAR\xspace}
\newcommand{\clcarfull}{\clcar: Conferencia Latinoamericana de Computación de Alto Rendimiento\xspace}

\newcommand{\nsf}{{\fontfamily{bookman}\sc NSF}\xspace}
\newcommand{\nsffull}{\nsf: National Science Foundation\xspace}

\newcommand{\conare}{{\fontfamily{bookman}\sc CONARE}\xspace}
\newcommand{\conarefull}{\conare: Consejo Nacional de Rectores\xspace}

\newcommand{\cenat}{{\fontfamily{bookman}\sc CeNAT}\xspace}
\newcommand{\cenatfull}{\cenat: Centro Nacional de Alta Tecnología\xspace}

\newcommand{\cnca}{{\fontfamily{bookman}\sc CNCA}\xspace}
\newcommand{\cncafull}{\cnca: Co-laboratorio Nacional de Computación Avanzada\xspace}
\newcommand{\cadejos}{{\fontfamily{bookman}\sc\textcolor{black}{cadejos}}\xspace}

\newcommand{\ictp}{ICTP: International Centre for Theoretical Physics\xspace}


\newcommand{\ucr}{Universidad de Costa Rica\xspace}
\newcommand{\eie}{Escuela de Ingeniería Eléctrica\xspace}
\newcommand{\fing}{Facultad de Ingeniería\xspace}

\newcommand{\towhom}{To Whom it May Concern\xspace}
\newcommand{\aquien}{A Quien Corresponda\xspace}

\newcommand{\fourVidTags}{\emph{4vid}\textit{Tags}\xspace}



%Conferences
\newcommand{\ines}{17th IEEE International Conference on Intelligent Engineering Systems -- INES\xspace}
\newcommand{\cinti}{14th IEEE International Symposium on Computational Intelligence and Informatics -- CINTI\xspace}
\newcommand{\sami}{IEEE International Symposium on Applied Machine Intelligence and Informatics -- SAMI\xspace}

%People
\newcommand{\juanD}{Juan Marcos Delgado Zumbado, MLE\xspace}
\newcommand{\juanDjob}{Director}
\newcommand{\juanDwho}{Estimado Sr. Delgado}

\newcommand{\lauraR}{Laura Robles Loaiza, Licda.\xspace}
\newcommand{\lauraRjob}{Unidad de adquisiciones, \os}
\newcommand{\lauraRwho}{Estimada Sra. Robles}

\newcommand{\yamilethF}{Yamileth Figueroa Barahona, MBA.\xspace}
\newcommand{\yamilethFjob}{Directora de la Dirección Financiera, Rectoría\xspace}
\newcommand{\olgaC}{Olga Cordero Quirós\xspace}
\newcommand{\olgaCjob}{Gestora de la Dirección Financiera, Rectoría\xspace}
\newcommand{\electrizarte}{ElectrizArte\xspace}

\newcommand{\oaice}{OAICE\xspace}
\newcommand{\oaicefull}{\oaice: Oficina de Asuntos Internacionales y Cooperación Externa\xspace}
\newcommand{\oaicename}{Oficina de Asuntos Internacionales y Cooperación Externa\xspace}

\newcommand{\julietaC}{Dra. Julieta Carranza Velázquez\xspace}
\newcommand{\julietaCjob}{Directora \oaice\xspace}
\newcommand{\julietaCwho}{Estimada Dra. Carranza\xspace}

\newcommand{\walterM}{Dr. Walter Marín Méndez\xspace}
\newcommand{\walterMjob}{Subdirector \oaicename\xspace}
\newcommand{\walterMwho}{Estimado Dr. Marín\xspace}

\newcommand{\fatimaA}{Fátima Acosta López\xspace}
\newcommand{\fatimaAjob}{Jefe de Sección de Movilidad Académica-Administrativa\xspace}
\newcommand{\fatimaAwho}{Estimada Sra. Acosta\xspace}

\newcommand{\haydeeR}{Haydée Ramos\xspace}
\newcommand{\haydeeRjob}{Académicos Visitantes, Becas Cortas\xspace}
\newcommand{\haydeeRwho}{Estimada Sra. Ramos\xspace}


\newcommand{\osg}{OSG\xspace}
\newcommand{\osgname}{Oficina de Servicios Generales\xspace}
\newcommand{\osgfull}{\osg: \osgname}

\newcommand{\oscarM}{M.Sc. Óscar Mario Molina Molina\xspace}
\newcommand{\oscarMjob}{Director\xspace}
\newcommand{\oscarMwho}{Estimado Sr. Molina\xspace}

\newcommand{\jeffreyD}{Ing. Jeffrey Dimarco Fernández\xspace}
\newcommand{\jeffreyDjob}{Jefe de la sección de transportes\xspace}
\newcommand{\jeffreyDwho}{Estimado Sr. Dimarco\xspace}

\newcommand{\audiP}{Audi Paniagua Gómez\xspace}
\newcommand{\audiPjob}{Sección de transportes\xspace}
\newcommand{\audiPwho}{Estimado Sr. Paniagua\xspace}




\newcommand{\anaAA}{Ana Alfaro Alvarado\xspace}
\newcommand{\anaAAjob}{Gestora de  Becas CU / Consejo de Rectoría / FDI / Membresías\xspace}
\newcommand{\anaAAwho}{Estimada Sra. Alfaro\xspace}

\newcommand{\MIT}{Massachusetts Institute of Technology\xspace}

\newcommand{\tec}{Tecnológico de Costa Rica\xspace}
\newcommand{\siplab}{SIP-Lab\xspace}

\newcommand{\ipcvlab}{IPCV-LAB\xspace}
\newcommand{\ipcvlabfull}{Image Processing and Computer Vision Laboratory\xspace}


\newcommand{\una}{Universidad Nacional\xspace}



\newcommand{\cieq}{Comisión Institucional de Equipamiento\xspace}
\newcommand{\odi}{Oficina de Divulgación e Información\xspace}
\newcommand{\proinnova}{PROINNNOVA\xspace}


\newcommand{\tecmo}{TECMO S.A.\xspace}

\newcommand{\citic}{CITIC\xspace}
\newcommand{\citicfull}{\citic: Centro de Investigación en Tecnologías de Información y Comunicación\xspace}

\newcommand{\cimohu}{CIMOHU\xspace}
\newcommand{\cimohufull}{\cimohu: Centro de Investigación en Movimiento Humano\xspace}

\newcommand{\ciet}{CIET\xspace}
\newcommand{\cietfull}{\ciet: Centro de Investigación en Enfermedades Tropicales\xspace}

\newcommand{\ciemic}{CIEMIC\xspace}
\newcommand{\ciemicfull}{\ciemic: Centro de Investigación en Estructuras Microscópicas\xspace}

\newcommand{\micitt}{MICITT\xspace}
\newcommand{\micittfull}{\micitt: Ministerio de Ciencia y Tecnología y Telecomunicaciones\xspace}

\newcommand{\estebanC}{Dr.~Esteban Chaves Olarte\xspace}

\newcommand{\alexandraM}{Dra.~Alexandra Martínez Porras\xspace}
\newcommand{\steveQ}{Dr.~rer.~nat.~Steve Quirós Barrantes\xspace}
\newcommand{\rodrigoM}{Dr.~rer.~nat.~Rodrigo Mora Rodríguez\xspace}

\newcommand{\ceciliaD}{Dra.~Cecilia Díaz Oreiro\xspace}
\newcommand{\ceciliaDjob}{Decana}
\newcommand{\ceciliaDwho}{Estimada Dra. Díaz}

\newcommand{\gabrielaM}{Dra.~Gabriela Marín Raventós\xspace}
\newcommand{\gabrielaMjob}{Directora}
\newcommand{\gabrielaMwho}{Estimada Dra. Marín\xspace}

\newcommand{\ramonB}{Ramón Bonilla Lizano\xspace}

\newcommand{\fernandoG}{Dr. Fernándo García Santamaría}
\newcommand{\cesarR}{Dr. César Rodríguez Sánchez}

\newcommand{\gabrielaB}{Dra.~Gabriela Barrantes Sliesarieva\xspace}

\newcommand{\ecci}{Escuela de las Ciencias de la Computación e Informática\xspace}

\newcommand{\ci}{Centro de Informática\xspace}

\newcommand{\aldebaran}{{\sc ALDEBARAN Robotics}\xspace}
\newcommand{\nao}{{\sc Nao}\xspace}
\newcommand{\robocup}{{\sc RoboCup}\xspace}

\newcommand{\nvidia}{{\sc NVIDIA Corporation}\xspace}

\newcommand{\naturalpoint}{{\sc Natural Point Inc.}\xspace}
\newcommand{\optitrack}{{\sc OptiTrack}\xspace}

\newcommand{\dell}{{\sc DELL}\xspace}

\newcommand{\samsung}{{\sc Samsung}\xspace}
\newcommand{\microexport}{{\sc Micro Export}\xspace}

\newcommand{\tum}{TUM\xspace}
\newcommand{\tumfull}{\tum: Technische Universität München\xspace}

\newcommand{\unibremen}{UniBremen\xspace}
\newcommand{\unibremenfull}{\unibremen: Universität Bremen\xspace}

\newcommand{\cotesys}{{\sc CoTeSys}\xspace}
\newcommand{\cotesysfull}{\cotesys: Cognition for Technical Systems\xspace}
\newcommand{\crlab}{{\sc CR-Lab}\xspace}
\newcommand{\crlabfull}{\crlab: Cognitive Robotics Research Laboratory\xspace}

\newcommand{\upgc}{Universidad de las Palmas de Gran Canaria\xspace}


\newcommand{\geovanniM}{Dr.-Ing.~Geovanni Martínez Castillo\xspace}
\newcommand{\geovanniMjob}{Coordinador Comisión de Investigación}
\newcommand{\geovanniMwho}{Estimado Don Geovanni}

\newcommand{\franciscoS}{Dr.~rer.~nat.~Francisco Siles Canales\xspace}
\newcommand{\franciscoSjob}{Coordinador del \prislab}
\newcommand{\franciscoSwho}{Estimado Don Francisco}


\newcommand{\jorgeR}{Jorge Romero Chacón, Ph.D.\xspace}
\newcommand{\jorgeRjob}{Director, \eie}
\newcommand{\jorgeRwho}{Estimado Don Jorge}

\newcommand{\randolphS}{Randolph Steinvorth Fernández, Ph.D.\xspace}
\newcommand{\randolphSjob}{Director}
\newcommand{\randolphSwho}{Estimado Don Randolph}

\newcommand{\eddieA}{Eddie Araya Padilla, Ph.D.\xspace}
\newcommand{\eddieAjob}{Coordinador de la Comisión de Credenciales, Currículum y Reconocimiento, \eie}
\newcommand{\eddieAwho}{Estimado Dr. Araya}


\newcommand{\edwinS}{Ing. Edwin Solórzano Campos, M.Sc.\xspace}
\newcommand{\edwinSjob}{Decano, \fing\xspace}
\newcommand{\edwinSwho}{Estimado Don Edwin\xspace}

\newcommand{\henningJ}{Dr.~Henning Jensen Pennington\xspace}
\newcommand{\henningJjob}{Rector\xspace}
\newcommand{\henningJwho}{Estimado Don Henning\xspace}

\newcommand{\eliecerU}{M.Sc. Eliécer Ureña Prado\xspace}
\newcommand{\eliecerUjob}{Director Consejo Universitario\xspace}

\newcommand{\jfranciscoA}{Ing. José Francisco Aguilar Pereira\xspace}
\newcommand{\jfranciscoAjob}{Representante Área de Ingeniería, Consejo Universitario\xspace}


\newcommand{\gloriaM}{Gloria Meléndez Celis, M.Sc.\xspace}
\newcommand{\gloriaMjob}{Directora Ejecutiva, Rectoría\xspace}
\newcommand{\gloriaMwho}{Estimada Doña Gloria\xspace}

\newcommand{\aliceP}{Alice L. Pérez Sánchez, Ph.D.\xspace}
\newcommand{\alicePjob}{Vicerrectora\xspace}
\newcommand{\alicePwho}{Estimada Dra. Pérez}

\newcommand{\carlosA}{Dr. Carlos Araya Leandro\xspace}
\newcommand{\carlosAjob}{Vicerrector\xspace}
\newcommand{\carlosAwho}{Estimado Dr. Araya}


\newcommand{\cristinaA}{Ana Cristina Alvarado Ulloa, Licda.\xspace}

\newcommand{\anaG}{Ana Isabel Gamboa Camacho, Br.\xspace}

\newcommand{\alonsoC}{Sr. Alonso Castro Vindas\xspace}
\newcommand{\alonsoCjob}{Analista Financiero, \vinv}
\newcommand{\alonsoCwho}{Estimado Sr. Castro\xspace}

\newcommand{\alonsoCM}{Alonso Castro Mattei, M.Sc.\xspace}
\newcommand{\alonsoCMjob}{Director}
\newcommand{\alonsoCMwho}{Estimado Sr. Castro\xspace}


\newcommand{\juanS}{Sr. Juan Manuel Sanabria\xspace}

\newcommand{\maurenR}{Sra. Mauren Reyes Umanzor\xspace}
\newcommand{\maurenRjob}{Unidad de Almacenamiento y Distribución\xspace}

\newcommand{\os}{Oficina de Suministros\xspace}
\newcommand{\vad}{Vicerrectoría de Administración\xspace}

\newcommand{\xiniaA}{Xinia Aguilar Sánchez, M.A.U.\xspace}
\newcommand{\xiniaAjob}{Jefe Administrativa\xspace}
\newcommand{\xiniaAwho}{Estimada Xinia\xspace}

\newcommand{\lochiY}{Lochi Yu Lo, Ph.D.\xspace}

\newcommand{\wajihaS}{Lcda. Wajiha Sasa Marín\xspace}
\newcommand{\ottoS}{Otto Salas Murillo\xspace}

\newcommand{\ieee}{{\sc IEEE}\xspace}
\newcommand{\iwobi}{{\sc IWOBI}\xspace}
\newcommand{\iwobifull}{\iwobi: International Conference and Workshop on Bioinspired Intelligence\xspace}
% \newcommand{\}{\xspace}


\newcommand{\maryhelenB}{Mary Helen Bialas\xspace}
\newcommand{\maryhelenBjob}{Academic Relations Manager\xspace}
\newcommand{\maryhelenBwho}{Estimada Mary Helen\xspace}



\newcommand{\jeffryD}{Ing. Jeffry Dimarco Fernández\xspace}
\newcommand{\jeffryDjob}{Jefe Sección de Transportes\xspace}

\newcommand{\teodoroW}{Teodoro Willink Castro\xspace}
\newcommand{\fabianA}{Fabián Abarca Calderón\xspace}
\newcommand{\diegoD}{Diego Dumani Jarquín\xspace}
\newcommand{\tonyO}{Tony Ortíz Salazar\xspace}

\newcommand{\juanC}{Dr.~Ing.~Juan Luis Crespo Mariño\xspace}
\newcommand{\juanCjob}{Profesor Investigador\xspace}
\newcommand{\juanCwho}{Estimado Dr. Crespo\xspace}

\newcommand{\saulC}{Saúl Calderón Ramírez\xspace}

\newcommand{\federicoR}{Dr.~rer.~nat.~in~fieri~Federico Ruiz Ugalde\xspace}
\newcommand{\federicoRjob}{Profesor Investigador\xspace}
\newcommand{\federicoRwho}{Estimado Sr. Ruiz\xspace}


\newcommand{\alvaroO}{Dr.~Álvaro de la Ossa Osegueda\xspace}
\newcommand{\alvaroOjob}{Director del Co-laboratorio Nacional de Computación Avanzada\xspace}
\newcommand{\alvaroOwho}{Estimado Dr. de la Ossa\xspace}


\newcommand{\certec}{CerTec~S.A.\xspace}
\newcommand{\orbe}{CEO~El Orbe\xspace}
\newcommand{\isc}{I.S. Corporación\xspace}

\newcommand{\sepucr}{SEP\xspace}
\newcommand{\sepname}{Sistema de Estudios de Posgrado\xspace}
\newcommand{\sepfull}{\sepucr: \sepname}

\newcommand{\feeii}{FEEII\xspace}
\newcommand{\feeiifull}{\feeii: Fondo de Estímulo Especial para la Investigación e Intersedes\xspace}


\newcommand{\vinv}{Vicerrectoría de Investigación\xspace}
\newcommand{\juanSanabria}{Sr.~Juan Manuel Sanabria Mora, Téc.~Vic.~Inv.\xspace}

\newcommand{\vadm}{Vicerrectoría de Administración\xspace}

\newcommand{\auge}{AUGE\xspace}
\newcommand{\augefull}{\auge: Agencia Universitaria para el Emprendimiento\xspace}
\newcommand{\ucrea}{UCREA\xspace}
\newcommand{\ucreafull}{\ucrea: Espacio de Estudios Avanzados de la \ucr}
\newcommand{\iia}{IIArte\xspace}
\newcommand{\iiafull}{\iia: Instituto de Investigaciones en Arte\xspace}



\newcommand{\fs}[1]{\fontsize{#1}{#1pt+0.2*#1pt}\selectfont} %font size

\newcommand{\dfg}{Deutsche Forshugnsgemainschaft\xspace}
\newcommand{\aspogamo}{ASpoGAMo\xspace}

\newcommand{\intelcr}{Componentes Intel de Costa Rica\xspace}

\hyphenation{e-du-ca-ti-va la-bo-ra-to-rios e-du-ca-ti-vos li-te-ra-tu-ra ca-pa-ci-ta-ción a-cep-ta-ción gra-dua-ción re-so-lu-ción rea-li-za co-la-bo-ra-ción in-ter-na-cio-na-les res-pe-tuo-sa-men-te do-cu-men-to con-si-de-rar an-te-rior-men-te cum-pli-mien-to re-que-ri-mien-to Mi-nis-te-rio au-to-ri-zar ac-tual me-dian-te si-guien-tes au-to-ma-ti-za-do e-va-lua-ción rea-li-zar-se es-tu-dian-te in-ter-dis-ci-pli-na-ria equi-pa-mien-to ge-ne-ra-dos de-sa-rro-lla-das Pos-doc-to-ran-te Re-co-no-ci-mien-to mi-sión de-sa-rro-llar pro-ble-mas res-pon-sa-ble in-quie-tu-des pro-pues-ta es-tu-dian-tes u-ni-da-des in-ves-ti-ga-do-res des-cri-tos ge-ne-ra ne-ce-sa-rios Es-pe-ra-mos co-rres-pon-dien-tes rea-li-zar an-te-rior co-la-bo-ra-cio-nes la-bo-ra-to-rio bi-blio-gra-fí-a ma-ne-ra cons-ta-ta vi-sua-li-za-ción si-mu-la-ción ti-tu-la-do des-cri-to ad-mi-nis-tra-ti-vas ca-rac-te-ri-zar ge-ne-ra-rá di-se-ño ge-ne-ra-ción mo-de-los rea-li-za-rán re-que-ri-mien-tos res-pon-sa-bi-li-da-des He-rra-mien-tas u-sua-rios va-
li-da-ción rea-li-za-das de-sa-rro-llo an-te-rio-ri-dad he-rra-mien-ta bi-blio-te-cas ins-truc-cio-nes des-pués co-rres-pon-den es-ta-ble-ci-mien-to cam-peo-na-to con-ti-nua-rá Vi-ce-rrec-to-ría re-fe-ren-cia me-dian-te co-rres-pon-dien-tes ca-rac-te-rís-ti-cas res-pues-ta re-so-lu-cio-nes a-pro-pia-da-men-te Ge-ren-te ins-tan-cias nues-tro par-ti-ci-pa-ción si-guien-te e-di-fi-cio par-ti-cu-lar man-te-ni-mien-to so-li-ci-tó}


\newcommand{\ra}[1]{\renewcommand{\arraystretch}{#1}}

\addto\captionsspanish{\renewcommand{\tablename}{Tabla}}					% Cambiar nombre a tablas
\addto\captionsspanish{\renewcommand{\listtablename}{Índice de tablas}}		% Cambiar nombre a lista de tablas

\newcolumntype{C}[1]{>{\centering\let\newline\\\arraybackslash\hspace{0pt}}m{#1}}

\begin{document}
	\frontmatter
	
	%título del proyecto
	\title{Implementación en Verilog de Unidad de Generacion de Rayos para GPU Theia.}
	
	%nombre completo del autor
	\autor{Josué David Vargas Amador}
	
	%fecha de la presentación oral
	\date{Diciembre de 2015}
	
	%tribunal evaluador
	%profesor guía
	\dca{MSc. Diego Valverde Garro}
	
	%miembros del tribunal (lectores)
	\maca{MSc. Carlos Duarte Martínez}
	\mbca{MSc. Rodolfo Brenes Fernández}
	
	%--------------------------------------------------------------------
	\eietitlepage
	\cleardoublepage 
	\eieaprovalpage
	\cleardoublepage
	
	%resumen
	%\begin{center}\huge{\textbf{Dedicatoria}}\end{center}

%Dedico este proyecto a las siguientes personas:
%\begin{itemize} 
%	\item A mis papás, que por ellos hago todo.
%	\item A mi familia por creer en mí.
%	\item A mis amigos por acompañarme en el camino.
%    \item A la vida que me ha dado tanto. 
%\end{itemize}

%\cleardoublepage

%\begin{center}\huge{\textbf{Reconocimientos}}\end{center}

%Agradezco a mi profesor guía, MSc. Diego Valverde Garro por darme la oportunidad de conocer su proyecto personal THEIA, mi trabajo se finalizó gracias a sus explicaciones y paciencia.

%\cleardoublepage

\begin{center}\huge{\textbf{Resumen}}\end{center}

%--------------------------------------------------------------------

En el siguiente documento se lleva a cabo la descripción de un código en lenguaje Verilog encargado de la generación de rayos que estará dentro de la arquitectura del proyecto de código libre llamado GPU THEIA. Este módulo debe poseer un conjunto de instrucciones de modo que pueda realizar los cálculos necesarios para la normalización de los vectores que llegan como entrada al GPU desde memoria, por ello debe poseer la capacidad de aproximar el valor del inverso de las raíces cuadradas.  

El conjunto de instrucciones de la Unidad de Generación de Rayos (\textit{RGU}, por sus siglas en inglés) implementa un método numérico para la aproximación del inverso de la raíz cuadrada después de obtener una primer aproximación proveniente de una tabla que almacena un número limitado de valores. El formato de los números que manipula la Unidad de Generación de Rayos es punto fijo.

Además se implementa un submódulo que realiza aproximaciones para valores superiores no presentes en la tabla, debido a la limitación del hardware, para ello se estudia el efecto de realizar múltiples iteraciones en la aproximación de los inversos de raíces cuadradas al obtener aproximaciones que al inicio se alejan considerablemente del valor esperado.

Por último se estudia el efecto de la variación de la coma en el formato de los números en punto fijo con la finalidad de obtener un diseño con porcentajes de error del orden de 0.001 \% en la generación de rayos.
	\cleardoublepage
	
	%--------------------------------------------------------------------
	%sección inicial
	\tableofcontents*  %índice general
	
	\clearpage
	\newpage
	\listoffigures  %índice de figuras
	
	\clearpage
	\newpage
	\listoftables   %índice de cuadros
	\cleardoublepage
	
	%nomenclatura
	\include{nomenclatura}

	%--------------------------------------------------------------------
	\mainmatter
	\pagestyle{eieheadings}
			
	
	%Objetivos
	\chapter{Introducción}

\section{Justificación}
Los sistemas computacionales actuales poseen, dentro de su arquitectura módulos de hardware especializados llamados Unidades de Procesamiento Gráfico (GPU, por sus siglas en inglés) encargados de acelerar el proceso de representación de objetos tridimensionales en la pantalla del computador.

Las unidades de procesamiento gráfico permiten la visualización de objetos mediante el cálculo de las primitivas que conforman el modelo abstracto de las imágenes. Las GPU implementan distintos algoritmos  de representación gráfica, entre estos, uno es el algoritmo de Ray Casting.

El algoritmo de Ray Casting genera vectores (rayos) normalizados desde la perspectiva del usuario y calcula la intersección de los rayos con los objetos del escenario, además colabora con la formación de los colores, con la finalidad de crear las imágenes mostradas en pantalla.

Entonces los cálculos para la representación de objetos visuales en una GPU de tipo raycasting requieren de una arquitectura interna capaz de la generación de vectores (rayos) normalizados,  para luego usar estas estructuras de datos en los módulos dedicados a la intersección de rayos. La generación de rayos requiere de instrucciones capaces de  realizar cálculos aritméticos como multiplicaciones, sumas y restas, así como operaciones especializadas que permitan aproximar los valores de raíces cuadradas, por lo que el diseño lógico de una unidad dedicada facilitaría el proceso de creación rayos y permitiría añadir flexibilidad y modularidad al diseño de todo el GPU. 

Dentro de las referencias encontradas se hallan proyectos de hardware relacionados al diseño de arquitecturas de trazado de rayos que implementan unidades de generación de rayos propias como SaarCor de la Universidad de Saarland (\cite{Schmittler2004}) y RayCore de la Universidad de Sejong (\cite{Nah2014}). 

En el caso de la GPU de tipo raycasting Theia, las especificaciones arquitectónicas indican la necesidad de una Unidad de Generación de Rayos (RGU, por sus siglas en inglés). La RGU debe poseer un conjunto de instrucciones necesarias para el cálculo de la normalización de vectores tridimensionales empleados en las siguientes etapas de funcionamiento del GPU.

\section{Alcances del proyecto}

La GPU Theia se encuentra en su tercera iteración, y en esta etapa tiene dos módulos principales dentro de su descripción de RTL en el lenguaje Verilog: la unidad de generación de rayos normalizados (RGU) y el módulo de intersección de rayos de tipo AABB (siglas en inglés de Axis Aligned Bounding Boxes).

El propósito del presente proyecto es la implementación conductual de una RGU cuya descripción posea las instrucciones necesarias para el funcionamiento apropiado de la generación de rayos normalizados. Estas instrucciones deben ser capaces de proveer la información necesaria para programar el módulo de la RGU de manera que permita el cálculo aproximado del inverso de la raíz cuadrada empleando el método iterativo para aproximación de raíces.

A partir del diseño inicial se deben buscar las condiciones apropiadas que permitan obtener una Unidad de Generación de Rayos que produzca resultados con porcentajes de error promedio inferior a 1 porciento respecto al valor real del inverso de las raíces cuadradas de los vectores. 
Posterior a esto se debe plantear un ambiente de verificación funcional que permita afirmar que el módulo RGU está cumpliendo con su papel dentro de la arquitectura y que puede generar la información requerida por los módulos de intersección de rayos.

\section{Objetivos}

\subsection{Objetivo general}
 Desarrollar el modelo por comportamiento en Verilog de una Unidad de Generación de Rayos de un GPU tipo ray casting.


\subsection{Objetivos específicos}
 Desarrollar el modelo por comportamiento en Verilog de una Unidad de Generación de Rayos de un GPU tipo ray casting.

\begin{itemize} % lista con viñetas
	\item Investigar bibliografía sobre el mecanismo generación de rayos.
	\item Definir el mecanismo de generación de rayos normalizados en el GPU.
	\item Verificar el comportamiento funcional de la Unidad de Generación de Rayos en el GPU.
\end{itemize}
 
\section{Metodología}

\begin{enumerate}  %lista numerada
	\item Se procederá a investigar los conceptos fundamentales de la arquitectura de la GPU, el algoritmo de ray casting y sobre los posibles mecanismos de la generación de rayos normalizados.
	\item Se buscará la implementación final de la arquitectura interna de la RGU de modo que contenga las instrucciones necesarias para la normalización.
	\item Se simulará la ejecución del código en la RGU para generar los rayos normalizados necesitados por los módulos de intersección de rayos de tipo AABB.
	\item Se verificará el comportamiento funcional del módulo RGU con la finalidad de establecer un marco de referencia para la futura validación del resto de la versión actual del GPU Theia.
\end{enumerate}


\section{Desarrollo}

Este proyecto se estructura por medio de capítulos, cada uno tiene como tarea aclarar los siguientes tópicos:

\begin{enumerate}

\item Capítulo I: Introducción. 
Muestra la justificación del proyecto, los alcances y limitaciones, los objetivos y la metodología que permite cumplir los mismos.

\item Capítulo II: Antecedentes y Marco Teórico. 
Introduce al lector conceptos claves de arquitectura de unidades de procesamiento gráfico, algoritmo de raycasting, y plantea los casos de proyectos donde se han implementado chips.

\item Capítulo III: Implementación final de la unidad de generación de rayos. Aquí se describe la arquitectura final de la unidad, así como el método empleado usando las instrucciones de ésta para implementar la unidad.

\item Capítulo IV: Prueba de verificación funcional. Se comprueba la funcionalidad del módulo conductual de lenguaje Verilog por medio de un ambiente de verificación apropiado.

\item Capítulo V: Conclusiones y recomendaciones. Se muestran posibles resultados del proyecto y reflexiones sobre el futuro del proyecto. 

\end{enumerate}
	
	%Marco Teorico
	
	\chapter{Marco Teórico}%\phantom{\cite{beetz09ijcss}}
%\textcolor{red}{En desarrollo!!!!}.

\section{Raycasting}

Se presentan detalles acerca del algoritmo de raycasting en el cual se basa el GPU Theia para su funcionamiento.

\subsection{Definición}

El algoritmo de raycasting funciona haciendo cálculos un píxel a la vez, y para cada píxel la tarea básica es encontrar el objeto que es observado en la posición correspondiente a ese píxel en la imagen. Se puede decir que cada píxel ve en una dirección distinta y cualquier objeto que es observado por un píxel debe intersectar el rayo proveniente desde el punto de vista de la cámara. El objeto esperado es aquel que es intersectado primero por el rayo más cercano a la cámara. Una vez que el objeto es encontrado, se emplea el punto de intersección, la superficie normal, y la otra información para definir el color de cada píxel. 

Entonces se puede decir que un algoritmo de raycasting tiene tres partes básicas:

\begin{enumerate}

\item Generación de rayo: donde se calcula el origen y la dirección de cada rayo (vector) del píxel correspoendiente en la vista de la cámara.
\item Intersección de rayo: donde se determina el objeto más cercano en la intersección del rayo proveniente de la cámara.
\item Shading: donde se calcula el color del píxel basado en los resultados de la intersección de rayos.

\end{enumerate}

Con el objetivo de generar rayos, primero se necesita una representación matemática de un rayo. Un rayo en realidad es solo un punto de origen y una dirección propagación, una línea paramétrica en 3D que va desde el ojo llamado punto e hasta otro punto s que está en el plano de la imagen está dada por: 

\begin{equation}
\label{eq:ray_definition}
  p(t) = e+t(s-e)
\end{equation}

Esta fórmula implica que se empieza en el punto e y se avanza a través del vector s-e hasta llegar al punto p. Valores negativos de t implica que se encuentra detrás del ojo.

Un seudocódigo sobre el algoritmo de raycasting es el siguiente:

\begin{enumerate}

\item For every pixel \# para cada píxel 
\item \hspace{5 mm} Construct a ray from the eye \hspace{3 mm} \# construya un rayo
\item \hspace{5 mm} For every object in the scene \hspace{3 mm} \# para cada objeto en la escena
\item \hspace{10 mm}	 Find intersection with the ray \hspace{3 mm} \# encuentre la intersección 	
\item \hspace{10 mm} Keep if closest 				\hspace{3 mm} \# Guarde el más cercano

\end{enumerate}

\section{GPU}
\subsection{Definición}

Las unidades de procesamiento gráfico se encargan de rápidamente renderizar (representar) objetos 3D en forma de píxeles en la pantalla de la computadora típicamente por medio de arquitecturas de hardware basadas en la técnica de rasterización. La mayor parte de las GPU han sido diseñadas para realizar operaciones fijas organizadas en forma de pipeline para ir pasando vértices y píxeles a través de distintas etapas.  

A continuación se mencionan las etapas principales del pipeline de gráficos:
\begin{enumerate}

\item El programa de usuario proporciona los datos al GPU en la forma de primitivas como puntos, líneas y polígonos que describen la geometría 3D. 
\item Etapa geométrica: las primitivas geométricas son procesadas en base a los vértices y son transformados de coordenadas 3D a triángulos 2D en la pantalla.. 
\item Etapa de rasterización: Los objetos en la pantalla pasan a los procesadores de píxeles y son rasterizados y coloreados para eventualmente mostrarse en el monitor.

\end{enumerate}
	
	%desarrollo
	\chapter{Desarrollo de la aplicación}

\section{Generalidades}

En esta sección se presentan generalidades del diseño de la aplicación. Se explica el por qué se siguió el modelo MVC tradicional y los navegadores con los cuales se garantiza que la aplicación sea completamente funcional.

\subsection{Modelo MVC}

El patrón de diseño MVC es uno de los más populares para la creación de aplicaciones, tanto de escritorio como web. Esto debido a que permite que los componentes de la arquitectura del software puedan variar de manera casi independiente y aún así seguir proveyendo una correcta funcionalidad a la aplicación. De esta manera, no solo se puede hacer un diseño limpio y fácil de entender y depurar, sino que es más fácil realizar actualizaciones y darles mantenimiento, ya que si algo falla, es bastante evidente cual de los componentes es el que está fallando. Además, el MVC facilita la programación orientada a objetos, paradigma que ya se tiene en HTML5 con los DOM, por lo que es apropiado utilizarlo. \\

Para seguir este patrón las divisiones se realizaron de la siguiente manera:

\begin{enumerate}
\item Los Views están compuestos por los diferentes archivos en el lenguaje HTML5 que le permiten al usuario visualizar en su navegador la aplicación y poder interactuar con la misma.

\item Los Controllers son varios archivos en JavaScript, utilizando el framework AngularJS, que permiten dar funciones específicas a las interacciones que tiene el usuario con los Views. Los controllers envían peticiones HTTP, como los son los POSTs, GETs, PUTs para modificar los contenidos del Model.

\item El Model se realizó utilizando el framework LoopBack de Stronloops y algunas funciones del NodeJS. Además la persistencia de los datos se realiza haciendo uso de una base de datos en MongoDb.  

\end{enumerate}

\subsection{Navegadores}

El diseño de aplicaciones web con lleva un reto: se debe de realizar tomando en cuenta las diferentes tecnologías que utilizan los navegadores comerciales más populares que se usan en la actualidad: Chrome, Firefox, Iceweasel, Opera, Internet Explorer de Microsoft (fuera de soporte y próximamente se utilizará Edge) y Safari en Mac OS X.\\

Para el desarrollo de la aplicación de este proyecto, se seleccionaron los dos navegadores más utilizados en la actualidad, Firefox y Chrome. Se diseñó y corroboró el funcionamiento en estos dos navegadores pero no se garantiza el completo funcionamiento de la aplicación en otros navegadores, con la excepción de Iceweasel, que es la versión de software libre de Firefox, presente en los sistemas operativos Debian, ya que para este navegador también se realizaron pruebas y funciona de igual manera que Firefox.

\section{Frontend}

Para el desarrollo del frontend de la aplicación se eligió utilizar HTML5, CSS3 y AngularJS. Esto debido a las funcionalidades predeterminadas que tiene HTML5 como cargar videos, realizar streaming de video y los elementos como el canvas. CSS3 es el que le brinda el estilo a los elementos del documentos permitiendo personalizar los colores, fuente, posiciones y otras características de la página web, dándole así la apariencia deseada. Finalmente Angular permite darle las características dinámicas que requiere la interfaz para obtener una buena experiencia de usuario.\\

\subsection{Consideraciones Generales}

Como una consideración general en todos los modos de operación de la aplicación, para operar sobre los cuadros del video se utiliza el elemento video de las librerías de HTML5 y se utiliza la propiedad del video \emph{currentTime} al cual se puede escribir para poner un valor de punto flotante en segundos para moverse a ese punto exacto del video. Sabiendo la duración del video en segundos y la tasa de FPS, se multiplican para obtener la cantidad de cuadros totales. Para avanzar cuadro por cuadro se le suma a \emph{currentTime} + $1/\text{FPS}$, para retroceder cuadro a cuadro se le resta, y para ir a un cuadro específico solo se iguala a $\frac{ \text{i}}{FPS}$, donde i es el número de cuadro al que se quiere ir, y con dicha división se obtiene el tiempo actual en segundos al que se quiere ir. 

En AngularJS, como se explicó en el marco teórico se tienen diferentes formas de llevar a cabo las funciones del fronted. Entre ellas están los controllers que son la principal fuente de interacción con el usuario y los services, que son ayudan a tener una funcionalidad general en todos los views de una aplicación. Por esta razón el menu principal se realizó como un service y los demás métodos de interacción con el usuarios se pusieron dentro de las clases de controllers.

La estructura del frontend cuenta con los siguientes componentes:

\begin{enumerate}
\item Controllers:
	\begin{enumerate}
		\item View1Ctrl: controller para el modo de operación de segmentación temporal.
		\item View2Ctrl: controller para el modo de operación de trayectorias de objetos.
		\item View3Ctrl: controller para el modo de operación de segmentación por contornos.
		\item View4Ctrl: controller para el modo de operación de segmentación semántica
	\end{enumerate}
\item Services:
	\begin{enumerate}
		\item LeftMenu: service presente en todos los modos de operación para tener acceso al menu principal de la aplicación.
	\end{enumerate}

\item App: módulo principal de la aplicación. Es el que realiza el routing de las peticiones realizadas al servidor y lo redirige al controller indicado, el cual actualizará el view de la manera indicada para que se visualice la información de manera correcta.

\end{enumerate}


Siguiendo el modelo MVC, en esta parte del diseño se abarcan los componentes tanto de los Views como los Controllers. Estos se dividen en las siguientes secciones:\\

\begin{itemize}
\item Menu principal: Service LeftMenu, es un controller en el patrón MVC.

\item Interfaces: son cuatro, y todas tienen elementos tanto de Views, como los son archivos HTML5 y CSS de cada uno, y tienen los Controllers respectivos para las funcionalidades que se exponen más adelante Las interfaces son las siguientes:

 \begin{enumerate}
		\item Interfaz para la segmentación temporal
		\item Interfaz para las trayectorias de objetos
		\item Interfaz para la segmentación de contornos 
		\item Interfaz para la segmentación semántica
\end{enumerate}


\end{itemize}

A continuación se describe el menu principal y cada una de estas interfaces de frontend.

\subsection{Menu Principal}

Para contar con un menú principal que no interfiriera con el espacio necesario para poder visualizar los videos, y realizar las anotaciones de una manera cómoda y ordenada; se decidió hacer un menú deslizante en la parte izquierda de la pantalla. El menú se puede mostrar u ocultar haciendo click sobre el nombre de la aplicación. En la Figura \ref{fig:menuFull} se muestra como se observaría la interfaz del programa con el menú principal abierto y en la Figura \ref{fig:mainmenu} se hace un acercamiento al menú para que se puede apreciar bien el contenido del mismo. 

Desde este menú se varía el modo de operación de la aplicación, por lo que cada vez que se acceda a él y se seleccione un modo diferente se cargan las configuraciones predeterminadas para dicho modo y la interfaz se adecua de manera apropiada.

\begin{figure}
	\includegraphics[width=\linewidth]{images/menuFull}
	\caption{Menú principal deslizante desde la izquierda} \label{fig:menuFull}
\end{figure}

\begin{figure}
	\includegraphics[width=0.35\linewidth]{images/mainmenu}
	\caption{Acercamiento del menú principal} \label{fig:mainmenu}
\end{figure}

Además cada una de las interfaces cuenta con los mismos botones superiores que realizan las funciones de cargar un video local, cargar un video que se haya hecho disponible desde el servidor, cargar un proyecto anterior o crear un nuevo proyecto. Toda está información de proyectos nuevos se almacena directamente en la base de datos del servidor. Cada uno de estos botones despliega las ventanas mostradas en la Figura \ref{fig:loads}.

\begin{figure}
	
	\centering
	\begin{subfigure}[t]{0.48\textwidth}
		\includegraphics[width=\textwidth]{images/videoL}
		\caption{Selección de videos locales}
		\label{fig:v1}
	\end{subfigure}
	~ %add desired spacing between images, e. g. ~, \quad, \qquad, \hfill etc. 
	%(or a blank line to force the subfigure onto a new line)
	\begin{subfigure}[t]{0.48\textwidth}
		\includegraphics[width=\textwidth]{images/videos}
		\caption{Selección de videos desde la base de datos}
		\label{fig:v2}
	\end{subfigure}
	~ %add desired spacing between images, e. g. ~, \quad, \qquad, \hfill etc. 
	%(or a blank line to force the subfigure onto a new line)
	\begin{subfigure}[t]{0.48\textwidth}
		\includegraphics[width=\textwidth]{images/proyectos}
		\caption{Creación de un proyecto nuevo}
		\label{fig:v3}
	\end{subfigure}
	~ %add desired spacing between images, e. g. ~, \quad, \qquad, \hfill etc. 
	%(or a blank line to force the subfigure onto a new line)
	\begin{subfigure}[t]{0.48\textwidth}
		\includegraphics[width=\textwidth]{images/proyectoL}
		\caption{Menú para manipular datos previos y descargar}
		\label{fig:v4}
	\end{subfigure}
	\caption{Cargar proyectos desde la base de datos}\label{fig:loads}
	
\end{figure}

Además para los modos segmentación de contornos y trayectoria de objetos, se incluyó un selector de color para poder variar los tonos con lo que se hacen las anotaciones espaciales. Este selector se puede visualizar en la Figura \ref{fig:colores}, y el mismo aparece cada vez que se va a agregar un nuevo seguimiento o un nuevo contorno a la lista del video.


\begin{figure}
	\includegraphics[width=0.7\linewidth]{images/colores}
	\caption{Selector de colores para los modos de trayectorias y contornos} \label{fig:colores}
\end{figure}

A continuación se describe las interfaces de los diferentes modos de operación de manera detallada.

\subsection{Interfaz de segmentación temporal}

El primero de los modos de operación es el de segmentación temporal. Su interfaz completa se puede apreciar en la Figura \ref{fig:tempFull}. En esta se visualiza el video en la parte izquierda de la ventana principal. A la derecha se tiene dos ventanas secundarias.\\

\begin{figure}
	\includegraphics[width=\linewidth]{images/tempFull}
	\caption{Interfaz de la segmentación temporal completa} \label{fig:tempFull}
\end{figure}

En la superior aparece el botón de añadir evento, el cual inicia una secuencia de eventos que le permite al usuario seleccionar el tipo de evento que ha presentando. Se inicia seleccionando el cuadro en el que ha ocurrido, luego se elige el tipo de transición, después el cuadro final de la transición y por último el tipo de escena que inicia luego de finalizada la transición. En el caso especial de un corte que el cuadro inicial y el final son el mismo, se omite el paso de seleccionar cuadro final. Finalmente se presenta al usuario un cuadro resumen con los datos que el mismo ingresó y con el botón de guardar, se añade el evento a la base de datos. En la capturas de pantalla organizadas en la Figura   \ref{fig:temptemp} se muestra el proceso descrito anteriormente de manera visual. En la imagen \ref{fig:temp1} es el primer paso luego de seleccionar añadir un evento. Se avanza a \ref{fig:temp2} y se agrega el tipo de transición, finalmente en \ref{fig:temp3} se elije el tipo de escena que se inició. En \ref{fig:temp4} se puede observar el resumen de datos que se le muestra al usuario antes de grabar el evento creado. \\

\begin{figure}
	
	\centering
	\begin{subfigure}[b]{0.7\textwidth}
		\includegraphics[width=\textwidth]{images/temp1}
		\caption{Selección de primer cuadro}
		\label{fig:temp1}
	\end{subfigure}
	~ %add desired spacing between images, e. g. ~, \quad, \qquad, \hfill etc. 
	%(or a blank line to force the subfigure onto a new line)
	\begin{subfigure}[b]{0.7\textwidth}
		\includegraphics[width=\textwidth]{images/temp2}
		\caption{Selección del tipo de transición}
		\label{fig:temp2} 
	\end{subfigure}
	~ %add desired spacing between images, e. g. ~, \quad, \qquad, \hfill etc. 
	%(or a blank line to force the subfigure onto a new line)
	\begin{subfigure}[b]{0.7\textwidth}
		\includegraphics[width=\textwidth]{images/temp3}
		\caption{Selección del tipo de datos antes de guardar}
		\label{fig:temp3}
	\end{subfigure}
	~ %add desired spacing between images, e. g. ~, \quad, \qquad, \hfill etc. 
	%(or a blank line to force the subfigure onto a new line)
	\begin{subfigure}[b]{0.7\textwidth}
		\includegraphics[width=\textwidth]{images/temp4}
		\caption{Resumen para el usuario antes de almacenar el dato.}
		\label{fig:temp4}	
	\end{subfigure}
	\caption{Detalle del funcionamiento en el módulo de agregar evento a la segmentación temporal}\label{fig:temptemp}
	
\end{figure}

En la inferior aparece una pequeña ventana de visualización de todos los eventos que han sido agregados en dicho proyecto. En esta ventana se puede tomar dos decisiones por cada evento: ir al cuadro inicial en el que ocurren, o eliminarlo de la base de datos. Dicha se ventana se puede apreciar le Figura \ref{fig:temp5} De esta manera se pueden agregar y eliminar eventos de un proyecto. Al final de la visualización de todos los eventos, se presenta un botón de descarga, el cual a la hora de presionarlo, descarga a la computadora local un archivo de formato JSON con todas las anotaciones realizadas hasta el momento.\\

\begin{figure}
	\includegraphics[width=0.7\linewidth]{images/temp5}
	\caption{Panel para eliminar segmentaciones previas o para ir a la segmentación seleccionada.} \label{fig:temp5}
\end{figure}


Para este modo de operación, el programa graba constantemente de manera automática. Esto porque se evita la perdida de datos en cualquier eventualidad y porque al poder añadir y borrar de manera sencilla, no afecta que el programa guarde y actualice la base de datos de manera automática.


\subsection{Interfaz de trayectorias}

El modo para describir trayectorias varía su interfaz de gran manera en comparación a la segmentación temporal, esto debido a que ahora no solo se tiene información en una dimensión (el tiempo). En este modo se tiene información en tres dimensiones: en el eje $x$ de la imagen, en el $y$ y en el $tiempo$. Por esta razón la complejidad y opciones de la interfaz incrementan. Ahora no se tienen los dos paneles a la derecha. El video toma el 100\% de su tamaño en la interfaz de la aplicación. Es decir, un video de 1920x1080 se desplegará con estas dimensiones sin importar la resolución actual de la pantalla. Se habilitan barras de desplazamiento para la completa visualización del video si la dimensión de este es mayor a la resolución del monitor. Esta decisión de diseño se tomó ya que si se está realizando seguimiento de objetos, lo mejor es que la precisión de pixeles sea grande, y la mejor manera para esto es que la proporción entre el video original y el que realmente se despliega en pantalla sea de 1 a 1. La interfaz general de este modo se aprecia en la Figura \ref{fig:objFull}.\\

\begin{figure}
	\includegraphics[width=\linewidth]{images/objFull}
	\caption{Interfaz de trayectorias completa} \label{fig:objFull}
\end{figure}

El menú para agregar anotaciones al video en este modo se detalla en la Figura \ref{fig:objobj}. En \ref{fig:obj1} se muestran los botones para agregar una nueva trayectoria, para eliminar la trayectoria actual, para cambiar la trayectoria actual y para salvar todo el contenido editado hasta el momento en la base de datos. En \ref{fig:obj2} se muestra como se selecciona con un drop menu el nombre de la trayectoria que se quiere continuar editando. Finalmente en la Figura \ref{fig:obj3} se muestra como se ve el video luego de que se agrega una punto de trayectoria a algún objeto.\\

Debido a su naturaleza, en este modo no se permite agregar más de un punto por cuadro. Ya que cada uno de estos representa un momento en el tiempo y un objeto no puede estar en dos posiciones a la vez. Por esta razón, el agregar un segundo punto en el mismo cuadro se bloquea a menos de que el el punto sea previamente eliminado. De igual manera, este modo tiene un botón grande en la parte inferior que permite poder descargar las trayectorias que han sido cargadas o que se han trazado en el proyecto actual.\\

A diferencia del modo de operación anterior, para grabar los cambios realizados en las trayectorias de los objetos, es necesario pulsar el botón de grabar. Esto debido a que se agrega un punto por cada cuadro, por lo que no se quiere estar escribiendo tantas veces a la base de datos. El usuario es responsable por guardar los avances antes de cerrar la aplicación. Si se crea una nueva trayectoria o se cambia a otra, en estos casos el programa si genera un autoguardado en la base de datos del contenido de la trayectoria que se está dejando antes de cargar o crear uno diferente.

\begin{figure}
	
	\centering
	\begin{subfigure}[b]{\textwidth}
		\includegraphics[width=\textwidth]{images/obj1}
		\caption{Menú para administrar las trayectorias}
		\label{fig:obj1}
	\end{subfigure}
	~ %add desired spacing between images, e. g. ~, \quad, \qquad, \hfill etc. 
	%(or a blank line to force the subfigure onto a new line)
	\begin{subfigure}[b]{0.18\textwidth}
		\includegraphics[width=\textwidth]{images/obj2}
		\caption{Selección de trayectoria actual}
		\label{fig:obj2}
	\end{subfigure}
	\qquad \qquad \qquad ~ %add desired spacing between images, e. g. ~, \quad, \qquad, \hfill etc. 
	%(or a blank line to force the subfigure onto a new line)
	\begin{subfigure}[b]{0.18\textwidth}
		\includegraphics[width=\textwidth]{images/obj3}
		\caption{Rastreo de la posición del objeto}
		\label{fig:obj3}
	\end{subfigure}
	\caption{Elementos de la interfaz para rastreo}\label{fig:objobj}
	
\end{figure}

\clearpage

\subsection{Interfaz de segmentación de contornos}

La segmentación espacial a partir de contornos comparte la misma interfaz que el modo anterior. Esto debido a que la visualización de información relevante es también en las dimensiones de $x$, $y$, y $t$. A diferencia del anterior, en este modo si se puede agregar más de un punto por cuadro, de hecho, la funcionalidad recae en que se tomen todos los pixeles por lo cuales se mueve el puntero del ratón mientras el botón izquierdo del mouse este presionado.

Como se puede apreciar en la Figura \ref{fig:contourFull} y comparando con la Figura \ref{fig:objFull} se ve que ambas interfaces son las mismas. Ahora bien, si se observa la Figura \ref{fig:contour1} se ve que el contorno realizado por el este modo de operación abarca todos los pixeles alrededor del objeto de interés. De igual manera se crean diferentes contornos, se puede elegir el color con el cual se realizará el contorno, se puede eliminar, se puede editar simplemente con salir, y volver al cuadro deseado y comenzar a dibujar de nuevo. Si se suelta el botón izquierdo del mouse mientras se realiza el contorno no supone ningún inconveniente, el contornos se puede retomar desde donde se dejó. Al finalizar el dibujo y avanzar al siguiente cuadro, la aplicación analiza los cuadros del contornos anterior y elimina del arreglo los pares ordenados que por alguna razón hayan quedado repetido, optimizando así el no guardar información que no sea útil.

\begin{figure}
	\includegraphics[width=\linewidth]{images/contourFull}
	\caption{Interfaz de segmentación de contornos completa} \label{fig:contourFull}
\end{figure}


\begin{figure}
	\includegraphics[width=0.2\linewidth]{images/contour1}
	\caption{Contorno de un jugador. Incluye todos los pixeles que conforman el contorno.} \label{fig:contour1}
\end{figure}


\subsection{Interfaz de segmentación semántica}

Finalmente se tiene el último modo de operación que es el de segmentación semántica. Esta interfaz es muy similar a la de segmentación temporal (como se puede apreciar en la Figura \ref{fig:semanticFull}), ya que en estos dos modos de operación el espacio de importancia es el tiempo. Al ser humano no le interesa tanto conocer las posiciones en $x$ o $y$ de los elementos en este modo de operación ya que la información semántica es obtenida a partir de método empíricos como la experiencia o la intuición. Lo que es relevante es que al presenciar cierto evento, se pueda capturar el inicio y final del mismo, y agregar una o más palabras clave que ayuden a clasificar el contenido. Por está razón la interacción de la interfaz del usuario se definió como se ve en la Figura \ref{fig:semantic1}.\\

Se puede observar que la acción que se le solicita al usuario es que presione los botones estando en el cuadro apropiado. Una vez que se utilizan botón de primer cuadro y último cuadro para fijar estos parámetros de correctamente solo hace falta introducir palabras clave separadas por espacios, que describan el contenido, las acciones o eventos que estén sucediendo en el video en un determinado momento. De esta manera se puede definir un período finito de tiempo con identificadores semánticos.\\

En la Figura \ref{fig:semantic2} se aprecia como son visualizados los datos semánticos previamente introducidos en la aplicación. Funciona de manera similar al modo de operación para segmentación temporal. Cada etiqueta posee dos botones, uno lo lleva al momento de inicio del evento, y el otro borra el evento de la base de datos.\\

\begin{figure}
	\includegraphics[width=\linewidth]{images/semanticFull}
	\caption{Interfaz de segmentación semántica completa} \label{fig:semanticFull}
\end{figure}

\begin{figure}
	
	\centering
	\begin{subfigure}[b]{0.7\textwidth}
		\includegraphics[width=\textwidth]{images/semantic1}
		\caption{Menu para agregar segmentaciones semánticas}
		\label{fig:semantic1}
	\end{subfigure}
	~ %add desired spacing between images, e. g. ~, \quad, \qquad, \hfill etc. 
	%(or a blank line to force the subfigure onto a new line)
	\begin{subfigure}[b]{0.7\textwidth}
		\includegraphics[width=\textwidth]{images/semantic2}
		\caption{Visualizador de segmentaciones semánticas actuales}
		\label{fig:semantic2}
	\end{subfigure}
\end{figure}

De esta manera se da por concluido el desarrollo de todos los modos de operación de la aplicación y como se ven y se utilizan sus respectivas interfaces.

\section{Backend}

Para el desarrollo del backend se utilizó la herramienta LoopBack. Esta provee una serie de funcionalidades genéricas para escribir y leer de bases de datos. Para configurar la herramientas e utilizan diferentes archivos JSON con los diferentes parámetros que definen el tipo de base de datos que se va a utilizar, las colecciones, y el modelo de los documentos. Los archivos .json de configuración se puede ver en la sección \ref{ap1} de los apéndices.\\

Una vez se hayan creado los archivos de configuración necesarios para tener cada una de las colecciones funcional, lo que se obtiene es un conjunto de métodos genéricos como los que se observan en la Figura \ref{fig:API}. Estos son diferentes tipos de métodos que se utilizan en las peticiones HTTP para realizar labores distintas. Lo métodos que se utilizaron en cada una de las colecciones fueron únicamente los POST, PUT, GET de la dirección raíz del servicio. Por ejemplo: un POST  a la dirección http:app@domain.com/api/contourSegmentation crearía un nuevo proyecto del tipo segmentación de contornos. Dichos métodos son suficientes para poder administrar correctamente lecturas y escrituras básicas a la base de datos. \\

Además de tener estos métodos genéricos, se puede tener métodos personalizados. Son desarrollados como si fuera un método del framework de NodeJS. Se creó un método especializado que puede ser utilizado por el administrador. Cuando este método se ejecuta, la función que realiza es escanear el directorio de videos dentro de la raíz del servidor, encuentra los archivos del tipo .mp4 y actualiza la base de datos para que dichos videos puedan estar disponibles para cargar desde el menu de carga desde el servidor remoto.\\

Finalmente la única configuración importante de cada uno de los modos de operación existentes que hace falta es la del modelo de los datos, es decir, como se ordena la información de cada uno de los documentos de la base de datos para que cada uno de ellos sea leído he interpretado de la manera correcta. Se tienen cuatro tipos específicos de modelo de datos y una forma general de como se almacenan los proyectos en la base de datos. Esto se explica con mayor detalle en la próxima sección que está dedicada a la base de datos.


\begin{figure}
	\includegraphics[width=\linewidth]{images/api}
	\caption{Visualización gráfica de los métodos del API} \label{fig:API}
\end{figure}

\section{Base de datos}


El diseño de la base de datos toma en cuenta cada uno de los diferentes modos de operación de la aplicación , por lo que se diseñó un tipo diferente de colección para cada uno de ellos. Todos los modos de operación tienen como base la estructura siguiente:

\begin{itemize}
\item \textit{id:} identificador único de la base de datos con el que se puede acceder a un documento. Es un número único en toda la colección para cada proyecto.
\item \textit{title:} variable en la cual se almacena el nombre del proyecto para luego poder cargarlo con facilidad, recordándolo por nombre.
\item \textit{author:} variable en la cual se almacena el autor del proyecto. 
\item \textit{data:} Es la información relevante de la segmentación realizada. Dependiendo del tipo de modo de operación, es la porción del documento de la base de datos que varía. Cada uno de los diferentes tipos se explican a continuación.
\end{itemize}

Como se observa la Figura \ref{fig:dbAr1}, todos los diferentes tipos de segmentación utilizan la misma estructura de proyecto. Lo que varía dentro de cada uno de ellos es la estructura del objeto denominado como \emph{data}. Las siguientes subsecciones se encargan de explicar el porqué de esta estructura.

\subsection{\textit{Data} en segmentación temporal}

En el caso de la segmentación temporal los datos relevantes son: cuadro inicial de una transición, cuadro final de una transición, el tipo de transición que ocurrió y el tipo de escena a la cual se llega luego de la transición. Por esta razón tiene el modelo de \emph{data} más sencillo de los cuatro modos de operación. Está compuesto por un arreglo de objetos donde los objetos tienen la siguiente estructura:\\

\textit{timeline data} = \{``initialFrame'',``lastFrame'',``eventType'',``sceneType''\}

\subsection{\textit{Data} en trayectorias de objetos}

Para este caso la información relevante es: cuadro actual, posición en X y posición en Y del centro del objeto en cada cuadro. Además a diferencia del modo de segmentación temporal, se puede tener más de un dato en el mismo período de tiempo, ya que durante cierto cuadro, pueden estar presentes dos o más objetos, a los cuales se debe de seguir de manera independiente. Por esta razón la estructura del dato es un poco más compleja y se puede observar como se compone en la Figura \ref{fig:dbAr1}.\\

Primero se asigna un nombre a la trayectoria, se le asigna un color específico que es con el que se van a visualizar las trayectorias en el video, y finalmente el apartado de data. En este se almacena un objeto que tiene tres partes, el número de cuadro de la posición en estudio, y las posiciones en $x$ y en $y$ del objeto. De esta manera se puede saber a cual objeto pertenece, donde está y en que instante de tiempo. Finalmente se puede representar como un objeto JSON de la siguiente manera:\\

\textit{track data} = \{``trackName'',``color'',``data''\}, donde data vendría dado por un arreglo de arreglos como el siguiente:\\

\textit{data} = [frameNumber,\{``xPosition'', ``yPosition''\}]

\subsection{\textit{Data} en segmentación espacial}

De manera muy similar se define el tipo de modelo para la segmentación de contornos. En este caso se aumentó un poco más el nivel de complejidad, ya que que los contornos no tienen únicamente una posición $x$ y $y$ como la posición, sino que más bien pueden tener N número de puntos. Por esta razón se extiende un poco más la profundidad del modelo y se hace un arreglo de objetos del tipo posición = \{``xPosition'',``yPosition''\} y finalmente se tiene la siguiente estructura, como se puede corroborar además en la Figura \ref{fig:dbAr1}:\\

\textit{contour data} = \{``contourName'',``color'',``contour information''\}\\

\textit{contour information} = [[``frameNumber'', ``contourPoints'' ]]\\

\textit{contourPoints} = [point0, ... , pointN],  donde pointX = \{``xPosition'',``yPosition''\}\\

\subsection{\textit{Data} en segmentación semántica}

Finalmente para la segmentación temporal se adopta un modelo muy similar al de la segmentación temporal, ya que no puede haber más una escena en el mismo espacio de tiempo, lo que se puede hacer es asignar una mayor cantidad de palabras clave al mismo segmento. El modelo para este tipo de segmentación queda definido de la siguiente manera:\\

\textit{semantic data} = \{``initialFrame'',``lastFrame'', keywords\}\\

\textit{keywords} = [``keyword0'', ..., ``keywordN'']\\


De esta manera se concluyen los cuatro tipos de modelos de la base de datos para los diferentes modos de operación según el tipo de segmentación que se quiera realizar.\\

Con esto se concluye el desarrollo de la aplicación y se continua con el análisis de resultados.

\begin{figure}
	\includegraphics[width=0.9\linewidth]{images/dbArq}
	\caption{Esquema del diseño de los modelos de la base de datos} \label{fig:dbAr1}
\end{figure}
	
	%resultados
	\chapter{Resultados}

\section{Prueba inicial revisada por señales de simulador}

En primera instancia es importante seleccionar señales que permitan verificar que el funcionamiento del dispositivo es el correcto, y que ejerciten las operaciones descritas anteriormente en la implementación del método de Newton-Raphson.

Se corrieron 32 pruebas con distintas variaciones en los bancos de prueba, donde se dividían 4 pruebas, para cada conjunto de iteraciones de 1 hasta 4 en el método de Newton-Raphson, relacionadas a un subconjunto de entrada que va de 8 bits hasta los 15 bits. Los bancos de prueba en la componente de chequeo guardaban los datos que luego se procesaron por medio de un programa en python que convirtió que calcula el error real que se había producido por medio de la Unidad de Generación de Rayos.

En esta sección del trabajo se muestran dos capturas de pantalla que corresponden a ciertas vistas del simulador que emplea el ISE de Xilinx, mostrando las etapas de una iteración del método de Newton-Raphson en el cálculo del inverso de la raíz cuadrada del número 2 en formato de punto fijo con escala 17.

En la imagen 1 se muestra en wA el valor de entrada correspondiente a 0x400000 que equivale a 2 en representación de punto fijo con la coma corrida 17 espacios hacia la izquierda. Se observa como en la primera instrucción el valor del registro de salida rResult toma el valor 0x16A09 que corresponde al valor de aproximación que provee la tabla de valores iniciales. De hecho el número 0x16A09 equivale en números reales aproximadamente a 0.7070999, esto se sabe al traducir 0x16A09 a decimal y después dividirlo entre $2^{17}$, ya en este caso la escala es igual a 17.  

En la tabla \ref{tab:nombres} se pueden apreciar información básica de las señales, con el objetivo de mejorar la comprensión de las señales vistas en el simulador.

% Table generated by Excel2LaTeX from sheet 'Hoja1'
\begin{table}[htbp]
  \centering
  \caption{Información básicas de señales en el simulador}
    \begin{tabular}{rr}
    \toprule
    Señales importantes & Importancia \\
    \midrule
    wA    & Valor del Registro A dentro del RGU \\
    wB    & Valor del Registro B dentro del RGU \\
    wDestination & Número de registro que se escribe en memoria  \\
    wOperation & Código que identifica cada instrucción \\
    rResult & Registro con el valor de salida del RGU \\
    \bottomrule
    \end{tabular}%
  \label{tab:nombres}%
\end{table}%

\begin{figure}
	\includegraphics[width=1\linewidth, height=2.5cm]{images/Selection_010}
	\caption{Primera parte de captura de la simulación de señales} \label{fig:sim1}
\end{figure}

En la imagen 1 se pueden seguir viendo las operaciones que corresponden a los pasos intermedios de multiplicación por los que pasa.

En la imagen 2 se observa que después de realizar las operaciones de multiplicación, resta y desplazamiento se obtiene que en la última operación el registro rResult adquiere el valor de 0x16A0A, equivalente a 0.7071075, lo cual implica que la iteración respecto al valor ini mejoró la aproximación del inverso de la raíz cuadrada pues el valor real es aproximadamente 0.7071067. Lo anterior implica la tendencia de mejorar la precisión siempre y cuando se encuentre un valor inicial cercano al valor meta. 

\begin{figure}
	\includegraphics[width=1\linewidth, height=2.5cm]{images/Selection_011}
	\caption{Segunda parte de captura de la simulación de señales} \label{fig:sim2}
\end{figure}

\section{Pruebas en los distintos rangos de bits de entrada}

El  RGU puede tener hasta 15 bits de parte entera pero tiene una tabla de memoria con 128 valores (7 bits) para generar una valor de iteración inicial, entonces se implementó un módulo llamado FixedPointSquareRoot cuya capacidad es aumentar el rango de cobertura de la unidad, pero dicho módulo aumenta el error al hacer una estimación ya que hace un corrimiento hacia la derecha de los 8 bits menos significativos de la parte entera de los números de entrada en formato de punto fijo para poder encontrar un valor en la tabla que abarca solo números de 7 bits (128 posibilidades).

Entonces sucede que existe una gran variabilidad respecto a los porcentajes de error dependiendo de los rangos mayores a 7 bits, por ejemplo los casos que son de 8 bits en la primera iteración provocan errores de casi el 50 porciento mientras que los casos de 12 bits poseen un error inferior al 1 porciento ante estímulos seudoaleatorios.

\begin{figure}
	\includegraphics[width=0.7\linewidth]{images/puntos}
	\caption{Gráfico sobre los porcentajes de error ante iteraciones con distintos rangos de bits} \label{fig:puntos}
\end{figure}

En el caso de la figura \ref{fig:puntos}, cuya escala en el eje Y es logarítmica, se puede observar, por ejemplo, que con tres iteraciones el comportamiento de las entradas con 10 bits de parte entera cambia mucho y para una cuarta iteración queda literalmente en la misma posición, lo cual implicaría que más de tres iteraciones en realidad no es necesario para obtener un margen de error apropiado para ese número de iteraciones.

En general la tendencia de las entradas con distintos bits es disminuir su porcentaje de error conforme se van aumentando las iteraciones. Solo en los casos de 14 y 15 bits se puede apreciar que se pierde el cáracter de disminución observado en los casos de menor cantidad de bits y más bien se genera un error casi constante, lo cual estaría dado por la incapacidad del hardware de proporcionar valores iniciales más precisos para lo cual se necesitaría tablas con mayor cantidad de valores.

En la tablas \ref{tab:errores1} y \ref{tab:errores2} se pueden observar que los porcentajes de error en 8 y 9 bits al iniciar las iteraciones eran bastante altos pero conforme se iteraba se alcanzaba un porcentaje de error más bajo.  

Lo que sucede es que para valores de solo 8 y 9 bits el error inicial es mucho ya que siempre se está haciendo un corrimiento de 8 bits a la parte entera del número. 

% Table generated by Excel2LaTeX from sheet 'Hoja1'
\begin{table}[htbp]
  \centering
  \caption{Tabla de iteraciones de 7 a 12 bits de entrada con su respectivo error}
    \begin{tabular}{rrrrrrrrr}
    \toprule
    Iteración     & 7     & 8     & 9     & 10    & 11    & 12\\
    \midrule
    1    & 0,01031246 & 47,7288136 & 31,2230569 & 13,3993714 & 6,78690633 & 0,86304167\\
    2    & 0,00627204 & 23,9909933 & 13,1553409 & 5,56002376 & 0,94484015 & 0,38862163\\
    3    & 0,00542083 & 5,05170113 & 2,36331935 & 0,09193496 & 0,16717202 & 0,34528176\\
    4    & 0,00393699 & 1,39280231 & 0,79338167 & 0,11496986 & 0,23323379 & 0,32996335\\
    \bottomrule
    \end{tabular}%
  \label{tab:errores1}%
\end{table}% 


% Table generated by Excel2LaTeX from sheet 'Hoja1'
\begin{table}[htbp]
  \centering
  \caption{Tabla de iteraciones de 7 a 12 bits de entrada con su respectivo error}
    \begin{tabular}{rrrrrrrrr}
    \toprule
    Iteración     & 13    & 14    & 15 \\
    \midrule
    1    &    0,77210039 & 1,26359914 & 3,47221991 \\
    2    &    0,98878366 & 1,35671632 & 3,47967437 \\
    3    &    1,0228597 & 1,65155714 & 4,43548373 \\
    4    &    0,89261429 & 1,42975391 & 4,55886896 \\
    \bottomrule
    \end{tabular}%
  \label{tab:errores2}%
\end{table}% 
%%%%%%%%%%%%%%%%%%%%%%%%%%%%%%%%%%%%%%%%%%%%%%%%%%%%%%%%%%%%%%%%%%%%%%%%%%%%%%%%%%%%%%%%%%%%%%%%%%%%%%%%%%%%
%%%%%%%%%%%%%%%%%%%%%%%%%%%%%%%%%%%%%%%%%%%%%%%%%%%%%%%%%%%%%%%%%%%%%%%%%%%%%%%%%%%%%%%%%%%%%%%%%%%%%%%%%%%%
%%%%%%%%%%%%%%%%%  MACHOTE  %%%%%%%%%%%%%%%%%%%%%%%%%%%%%%%%%%%%%%%%%%%%%%%%%%%%%%%%%%%%%%%%%%%%%%%%%%%%%%%%
%%%%%%%%%%%%%%%%%%%%%%%%%%%%%%%%%%%%%%%%%%%%%%%%%%%%%%%%%%%%%%%%%%%%%%%%%%%%%%%%%%%%%%%%%%%%%%%%%%%%%%%%%%%%
%%%%%%%%%%%%%%%%%%%%%%%%%%%%%%%%%%%%%%%%%%%%%%%%%%%%%%%%%%%%%%%%%%%%%%%%%%%%%%%%%%%%%%%%%%%%%%%%%%%%%%%%%%%%
\begin{comment}


\section{Pruebas para comprobar la funcionalidad y estabilidad}

Al finalizar el desarrollo de la aplicación, se probó acceder a ella desde computadoras en diferentes sistemas operativos. En todos los casos la aplicación operaba de forma correcta. Por lo que independientemente del OS del sistema, la herramienta es funcional, dando como resultado una aplicación web, multiplataforma. Lo cual era uno de los objetivos que se quería alcanzar inicialmente.\\

Además se analizó la exactitud de los datos obtenidos. A la hora de realizar los contornos con la herramienta desarrollado, siempre se obtuvo todos los pixeles que forman parte del mismo, distinto a Sensarea que se obtenían muy pocos puntos si se realizaba la funcionalidad de esta misma manera (dejando precionado el click izquierdo y desplazando el puntero al rededor del contorno).


\section{Pruebas realizadas por diversos usuarios}

Para probar que la aplicación diseñada en efecto es más sencilla de utilizar que las otras herramientas actuales y que con ella se generan datos de manera más eficiente, se procedió a realizar unas pruebas con tiempo a tres diferentes usuarios y se promedió el resultado. Las otras dos herramientas con las cuales se comparó fueron Sensara y VideoANT, debido a que son las únicas que no se requiere de conocimiento técnico para poder instalarlas o utilizarlas en línea. A cada uno se les solicitó realizar las siguientes tareas en orden:

\begin{enumerate}
\item Instalar o acceder a la aplicación
\item Abrir la aplicación, cargar uno de sus video y realizar la segmentación temporal de 5 escenas diferentes.
\item Seguir de la trayectoria de 5 elementos diferentes por al menos 20 cuadros.
\item Dibujar los contornos de 5 elementos diferentes por al menos 20 cuadros.
\item Realizar anotaciones semánticas de 5 tipos diferentes de escenas encontradas.
\end{enumerate}

El video que se utilizó fue una sección de la final de la Copa del Mundo Fifa 2010 entre España y Holanda. El peso del video es de 250 Mb, el formato es mp4 y codec es h264. Los resultados obtenidos se muestran en las tablas \ref{table:results} y \ref{table:results2}. La primera de ellas muestra lo que se duró haciendo la labor por primera vez y en la segunda tabla se muestra el valor promedio de tiempos en las 5 tareas de cada tipo.


\begin{table}[h]\centering
	
	\ra{2}
	\caption{Promedio de tiempo estimado de la primera función realizada correctamente}
	\label{table:results}
	
	\begin{tabular}{@{}cC{3cm}C{3cm}C{3cm}C{3cm}c@{}}\toprule
		
		& Función & GT-Tool & Sensarea & VideoANT&\\ \midrule
		
		& Primer uso de la aplicación & 7 segundos* & 2 minutos & 5 segundos* & \\
		
		& Segmentación temporal  & 26 segundos & 87 segundos & 15 segundos & \\
		
		& Rastreo de objetos & 5 segundos & 5 segundos & no aplica & \\
		
		& Segmentación de contornos & 27 segundos  & 34 segundos & no aplica & \\
		
		& Segmentación semántica & 20 segundos  & no aplica & 15 segundos & \\ 
		
		\bottomrule
		
	\end{tabular}
	
\end{table}

*Ese el tiempo que les tomó acceder a la página web, de lo contrario es el tiempo de descarga e instalación.\\

Se puede apreciar de esta primera tabla (\ref{table:results}), que GT-Tool es muy similar a Sensarea cuando se quiere realizar el seguimiento de la trayectoria de los objetos. Y es un poco más lento que VideoANT a la hora de realizar la segmentación temporal y semántica, esto es debido a la forma en la que VideoANT solicita la información, es más directa pero no tan especializada. De nuevo, VideoANT da precisión de segundos y no de cuadros y además no tiene manera de realizar seguimiento de trayectorias ni segmentación de contornos.\\

A continuación el promedio de las 5 repeticiones luego de haber realizado la tarea por primera vez:


\begin{table}[h]\centering
	
	\ra{2}
	\caption{Promedio de tiempo que tomó realizar cada función repetidas veces (5 repeticiones)}
	\label{table:results2}
	
	\begin{tabular}{@{}cC{3cm}C{3cm}C{3cm}C{3cm}c@{}}\toprule
		
		& Función & GT-Tool & Sensarea & VideoANT&\\ \midrule
		
		& Segmentación temporal  & 19 segundos & 30 segundos & 13 segundos & \\
		
		& Rastreo de objetos & 3 segundos & 4 segundos & no aplica & \\
		
		& Segmentación de contornos & 14 segundos  & 29 segundos & no aplica & \\
		
		& Segmentación semántica & 15 segundos  & no aplica & 13 segundos & \\ 
		
		\bottomrule
		
	\end{tabular}
	
\end{table}

Al analizar los resultados de dicha tabla se concluye los siguiente:

\begin{enumerate}
	
\item A medida que utilizan cualquier herramienta, el tiempo que les toma realizar una labor disminuye, sin importar cual sea. La disminución más grande la tuvo Sensarea en la segmentación temporal, pero esto fue debido a que, aunque Sensarea no soporta nativamente algún tipo de segmentación temporal, se puede simular colocando puntos con etiquetas. Cuando los usuarios se dieron cuenta de esto, lo empezaron a hacer y así hacían un tipo de segmentación temporal.

\item VideoANT continua siendo un poco más rápida para lo temporal y semántico, pero no cuenta con los otros tipos. De igual manera la diferencia en tiempos no es muy significativa.

\item GT-Tool si logra disminuir el tiempo en el que se generan los datos para la trayectoria de objetos y segmentación de contornos, mientras que a la vez aumenta la precisión brindada por Sensarea.

\end{enumerate}

\end{comment}
	
	%conclusiones
	\chapter{Conclusiones y recomendaciones}

Para finalizar este informe escrito, se expone las conclusiones, se da algunas recomendaciones y se propone trabajo que queda aún por realizar para seguir perfeccionando la herramienta. Las conclusiones se realizan con relación directa a los objetivos tanto el general como los específicos propuestos al inicio del trabajo y los resultados obtenidos luego de la creación de la aplicación y el uso de la misma.

\section*{Conclusiones}

\begin{itemize}
	
\item Se concluyó satisfactoriamente la creación de una aplicación web para la generación de los 4 diferentes tipos de segmentación requeridos: temporal, trayectorias, contornos y semántica. La aplicación es completamente funcional, permite cargar videos locales, guardar y cargar proyectos, y descargar los datos generados en formato JSON para poder procesarlos y compararlos con los datos que son obtenidos de los algoritmos.

\item El cargar videos desde el servidor se vio limitado a la funcionalidad que tienen los navegadores en la actualidad. Estos no están hechos para cargar todo el video rápidamente y poder manipular su tiempo tan deliberadamente, los videos en \emph{streaming} en los navegadores están diseñados para utilizar el mínimo ancho de banda posible y la principal función del navegador es reproducirlo una sola vez, por lo que no carga todo como es necesario y además va limpiando el \emph{buffer} de carga, por lo que los datos que se tenían cargados previamente, si ingresan muchos datos, son desechados y no se puede regresar a esas secciones del video sin volver a realizar un \emph{buffering} desde cero.

\item Las herramientas web disponibles en la actualidad satisfacen diferentes necesidades y diferentes gustos según el programador. En distintos lenguajes de programación como en Python y JavaScript, con diferentes tipos de servidores sincrónicos o asincrónicos. Quedará siempre a criterio del ingeniero que tipo de herramienta le es más útil dados sus gustos y el tipo de aplicación que quiere llegar a tener al final.

\item Se diseñó correctamente una base de datos del tipo NO-SQL, utilizando MongoDb. Esta base de datos tiene la ventaja de que escala rápidamente, que su formato de lectura y escritura predeterminado es JSON, el cual se utiliza ampliamente en los frontend en JavaScript o Python, lo que convierte a MongoDb en una excelente opción en bases de datos no relacionadas para aplicaciones o sitios web que se basen en estos lenguajes de programación.

\item Los resultados de comparar el uso de GT-Tool con Sensarea y VideoANT comprueban que se logró realizar una herramienta más eficiente en todos los aspectos a evaluar. Los datos de trayectoria y contorno se generan más rápido y con una resolución de puntos mayor para poder validar de mejor manera los algoritmos. En el caso de la segmentación temporal y semántica no es más rápida que VideoANT, pero esto se compensa ya que GT-Tool si logra realizar esta segmentación a nivel de cuadros y no de segundos como lo hace VideoANT. Además GT-Tool tiene de forma nativa e intuitiva el como realizar cada uno de los tipos de segmentación, y es una solución multiplataforma.

\item Se creó satisfactoriamente un manual básico y fácil de entender para que los usuarios puedan instalar y hacer uso de la herramienta correctamente y en poco tiempo. Además el video tutorial está disponible en línea en  \url{pris.eie.ucr.ac.cr/tools/gt-tool}.

\end{itemize}

\section*{Recomendaciones}

\begin{itemize}
	
	\item El desarrollo de software, especialmente de aplicaciones completas, es una tarea bastante complicada, más si todas las labores las realiza una sola persona. Se recomienda tener un equipo de más personas debido a que hacer el diseño de la interfaz gráfica, el desarrollo de funcionalidades y la verificación de las mismas, todo por una sola persona, con lleva mucho tiempo y no es lo ideal, ya que no se toma en cuenta lo que piensan otras persona de la interfaz y algunos errores se pasan por alto porque otra persona usó exhaustivamente la aplicación.
	
	\item En este tipo de diseños es importante asegurar que el código es legible para poder darle mantenimiento en presencia de cualquier problema. Es importante llevar una buena documentación a lo largo de todo el desarrollo ya que ayuda a mantener el orden y a entender mejor las labores que se está realizando.
	
	\item Seguir patrones de diseño como el MVC es de gran utilidad en el desarrollo de aplicaciones. Le da características modulares al código, por lo que si se quiere cambiar la interfaz del programa, se puede alterar únicamente los valores y archivos de los objetos que tienen que ver con los Views y los demás pueden mantenerse igual. Si todo se realiza correctamente la compatibilidad entre la nueva interfaz y el código de controllers y model anterior debe de funcionar en su totalidad o por lo menos con una compatibilidad muy alta.
	
\end{itemize}

\section*{Trabajo por realizar}

En vista de que un software nunca alcanza su versión final, siempre hay errores que se pueden reparar, desempeño que se puede mejorar o funcionalidades que se pueden agregar, se tiene una lista de funcionalidades o extras que pueden ser muy útiles de tener en la aplicación:

\begin{itemize}
\item Una plantilla en formato tipo JSON que se pueda cargar a los diferentes modos para personalizar el nombre de las etiquetas de la segmentación temporal o configurar previamente la cantidad de contornos o trayectorias y sus colores en los respectivos modos de operación.

\item Desarrollar alguna solución para poder cargar videos del servidor en su totalidad sin perder el buffering anterior o poder adelantarlo si se quiere cargar solo una sección adelantada del video.

\item La implementación de comandos comunes del teclado utilizados en la mayoría de software. Por ejemplo ctrl-z para deshacer y ctrl-s para guardar.

\item Crear una aplicación para dispositivos móviles que utilice la misma base de datos.

\item Instalar la aplicación en un servidor y realizar pruebas de estrés y carga por cantidad de usuarios y peticiones.

\item Ofrecer la posibilidad de descargar los datos generados en diferentes formatos: JSON, YAML, XML, entre otros.

\end{itemize}
	
	%--------------------------------------------------------------------
	%bibliografía
	\bibliography{eieclases}
		
	\cleardoublepage
	
	%--------------------------------------------------------------------
	%apéndice
	\appendix
	
	%\include{apendice0}
	%\include{apendice1}
	%\include{apendice2}
	
	%final del documento
\end{document}
