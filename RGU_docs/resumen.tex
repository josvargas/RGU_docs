%\begin{center}\huge{\textbf{Dedicatoria}}\end{center}

%Dedico este proyecto a las siguientes personas:
%\begin{itemize} 
%	\item A mis papás, que por ellos hago todo.
%	\item A mi familia por creer en mí.
%	\item A mis amigos por acompañarme en el camino.
%    \item A la vida que me ha dado tanto. 
%\end{itemize}

%\cleardoublepage

%\begin{center}\huge{\textbf{Reconocimientos}}\end{center}

%Agradezco a mi profesor guía, MSc. Diego Valverde Garro por darme la oportunidad de conocer su proyecto personal THEIA, mi trabajo se finalizó gracias a sus explicaciones y paciencia.

%\cleardoublepage

\begin{center}\huge{\textbf{Resumen}}\end{center}

%--------------------------------------------------------------------

En el siguiente documento se lleva a cabo la descripción de un código en lenguaje Verilog encargado de la generación de rayos que estará dentro de la arquitectura del proyecto de código libre llamado GPU THEIA. Este módulo debe poseer un conjunto de instrucciones de modo que pueda realizar los cálculos necesarios para la normalización de los vectores que llegan como entrada al GPU desde memoria, por ello debe poseer la capacidad de aproximar el valor del inverso de las raíces cuadradas.  

El conjunto de instrucciones de la Unidad de Generación de Rayos (\textit{RGU}, por sus siglas en inglés) implementa un método numérico para la aproximación del inverso de la raíz cuadrada después de obtener una primer aproximación proveniente de una tabla que almacena un número limitado de valores. El formato de los números que manipula la Unidad de Generación de Rayos es punto fijo.

Además se implementa un submódulo que realiza aproximaciones para valores superiores no presentes en la tabla, debido a la limitación del hardware, para ello se estudia el efecto de realizar múltiples iteraciones en la aproximación de los inversos de raíces cuadradas al obtener aproximaciones que al inicio se alejan considerablemente del valor esperado.

Por último se estudia el efecto de la variación de la coma en el formato de los números en punto fijo con la finalidad de obtener un diseño con porcentajes de error del orden de 0.001 \% en la generación de rayos.